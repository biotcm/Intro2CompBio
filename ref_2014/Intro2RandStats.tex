\documentclass[ignorenonframetext,]{beamer}
\usepackage{amssymb,amsmath}
\usepackage{ifxetex,ifluatex}
\usepackage{fixltx2e} % provides \textsubscript
\usepackage{lmodern}
\ifxetex
  \usepackage{fontspec,xltxtra,xunicode}
  \defaultfontfeatures{Mapping=tex-text,Scale=MatchLowercase}
  \newcommand{\euro}{€}
\else
  \ifluatex
    \usepackage{fontspec}
    \defaultfontfeatures{Mapping=tex-text,Scale=MatchLowercase}
    \newcommand{\euro}{€}
  \else
    \usepackage[T1]{fontenc}
    \usepackage[utf8]{inputenc}
      \fi
\fi
\IfFileExists{upquote.sty}{\usepackage{upquote}}{}
% use microtype if available
\IfFileExists{microtype.sty}{\usepackage{microtype}}{}
\usepackage{color}
\usepackage{fancyvrb}
\newcommand{\VerbBar}{|}
\newcommand{\VERB}{\Verb[commandchars=\\\{\}]}
\DefineVerbatimEnvironment{Highlighting}{Verbatim}{commandchars=\\\{\}}
% Add ',fontsize=\small' for more characters per line
\usepackage{framed}
\definecolor{shadecolor}{RGB}{248,248,248}
\newenvironment{Shaded}{\begin{snugshade}}{\end{snugshade}}
\newcommand{\KeywordTok}[1]{\textcolor[rgb]{0.13,0.29,0.53}{\textbf{{#1}}}}
\newcommand{\DataTypeTok}[1]{\textcolor[rgb]{0.13,0.29,0.53}{{#1}}}
\newcommand{\DecValTok}[1]{\textcolor[rgb]{0.00,0.00,0.81}{{#1}}}
\newcommand{\BaseNTok}[1]{\textcolor[rgb]{0.00,0.00,0.81}{{#1}}}
\newcommand{\FloatTok}[1]{\textcolor[rgb]{0.00,0.00,0.81}{{#1}}}
\newcommand{\CharTok}[1]{\textcolor[rgb]{0.31,0.60,0.02}{{#1}}}
\newcommand{\StringTok}[1]{\textcolor[rgb]{0.31,0.60,0.02}{{#1}}}
\newcommand{\CommentTok}[1]{\textcolor[rgb]{0.56,0.35,0.01}{\textit{{#1}}}}
\newcommand{\OtherTok}[1]{\textcolor[rgb]{0.56,0.35,0.01}{{#1}}}
\newcommand{\AlertTok}[1]{\textcolor[rgb]{0.94,0.16,0.16}{{#1}}}
\newcommand{\FunctionTok}[1]{\textcolor[rgb]{0.00,0.00,0.00}{{#1}}}
\newcommand{\RegionMarkerTok}[1]{{#1}}
\newcommand{\ErrorTok}[1]{\textbf{{#1}}}
\newcommand{\NormalTok}[1]{{#1}}
\usepackage{url}
\usepackage{letltxmacro}
\makeatletter
\def\maxwidth{\ifdim\Gin@nat@width>\linewidth\linewidth\else\Gin@nat@width\fi}
\def\maxheight{\ifdim\Gin@nat@height>\textheight0.8\textheight\else\Gin@nat@height\fi}
\makeatother
\AtBeginDocument{
  \LetLtxMacro\Oldincludegraphics\includegraphics
  \renewcommand{\includegraphics}[2][]{%
    \Oldincludegraphics[#1,width=\maxwidth,height=\maxheight,keepaspectratio]{#2}}
}

% Comment these out if you don't want a slide with just the
% part/section/subsection/subsubsection title:
\AtBeginPart{
  \let\insertpartnumber\relax
  \let\partname\relax
  \frame{\partpage}
}
\AtBeginSection{
  \let\insertsectionnumber\relax
  \let\sectionname\relax
  \frame{\sectionpage}
}
\AtBeginSubsection{
  \let\insertsubsectionnumber\relax
  \let\subsectionname\relax
  \frame{\subsectionpage}
}

\setlength{\parindent}{0pt}
\setlength{\parskip}{6pt plus 2pt minus 1pt}
\setlength{\emergencystretch}{3em}  % prevent overfull lines
\setcounter{secnumdepth}{0}

\title{生物信息导论}
\author{Songpeng Zu}
\date{Tuesday, October 07, 2014}

\begin{document}
\frame{\titlepage}

\begin{frame}
\tableofcontents[hideallsubsections]
\end{frame}

\section{Intro 2 R}\label{intro-2-r}

\begin{frame}{Content}

\begin{itemize}
\itemsep1pt\parskip0pt\parsep0pt
\item
  R Basic\\ 1.Install R and Rstudio\\ 2.R classes\\ 3.R statements\\ 4.R
  functions and libraries\\ 5.\textbf{R I/O}
\item
  R graph\\ 1.Basic graph\\ 2.ggplot\\ 3.\textbf{lattice}\\
\item
  \textbf{Others}\\ 1.\textbf{Latex vs R markdown}\\ 2.\textbf{Rcpp}\\
  3.\textbf{R shiny}\\ 4.\textbf{R regular expression}
\end{itemize}

\end{frame}

\begin{frame}[fragile]{R Basic}

\begin{itemize}
\itemsep1pt\parskip0pt\parsep0pt
\item
  Install R\\\url{http://www.r-project.org/}\\
\item
  Rstudio is a powerful IDE for R.\\\url{http://www.rstudio.com/}\\
\item
  Working directory
\end{itemize}

\begin{Shaded}
\begin{Highlighting}[]
\KeywordTok{getwd}\NormalTok{()}
\KeywordTok{setwd}\NormalTok{(}\StringTok{"D:/"}\NormalTok{)}
\end{Highlighting}
\end{Shaded}

\end{frame}

\begin{frame}[fragile]{R classes}

character, numeric, integer, logical, \textbf{vector, matrix, factor,
data frame, list.}

\begin{itemize}
\itemsep1pt\parskip0pt\parsep0pt
\item
  vector
\end{itemize}

\begin{Shaded}
\begin{Highlighting}[]
\NormalTok{x <-}\StringTok{ }\KeywordTok{c}\NormalTok{(}\DecValTok{10}\NormalTok{,}\DecValTok{3}\NormalTok{,}\DecValTok{2}\NormalTok{,}\DecValTok{1}\NormalTok{,}\OtherTok{NA}\NormalTok{); x[}\DecValTok{4}\NormalTok{]; }\KeywordTok{mode}\NormalTok{(x); }\KeywordTok{length}\NormalTok{(x) }
\end{Highlighting}
\end{Shaded}

\begin{itemize}
\itemsep1pt\parskip0pt\parsep0pt
\item
  matrix
\end{itemize}

\begin{Shaded}
\begin{Highlighting}[]
\NormalTok{X <-}\StringTok{ }\KeywordTok{matrix}\NormalTok{(}\DecValTok{1}\NormalTok{:}\DecValTok{12}\NormalTok{,}\DataTypeTok{nrow=}\DecValTok{3}\NormalTok{,}\DataTypeTok{byrow=}\OtherTok{TRUE}\NormalTok{,}\DataTypeTok{dimnames=}\OtherTok{NULL}\NormalTok{);X[}\DecValTok{1}\NormalTok{,]}
\NormalTok{## [1] 1 2 3 4}
\KeywordTok{rbind}\NormalTok{(X,}\KeywordTok{c}\NormalTok{(}\DecValTok{3}\NormalTok{,}\DecValTok{2}\NormalTok{,}\DecValTok{1}\NormalTok{,}\DecValTok{3}\NormalTok{))}
\NormalTok{##      [,1] [,2] [,3] [,4]}
\NormalTok{## [1,]    1    2    3    4}
\NormalTok{## [2,]    5    6    7    8}
\NormalTok{## [3,]    9   10   11   12}
\NormalTok{## [4,]    3    2    1    3}
\KeywordTok{apply}\NormalTok{(X,}\DecValTok{2}\NormalTok{,mean)}
\NormalTok{## [1] 5 6 7 8}
\end{Highlighting}
\end{Shaded}

\end{frame}

\begin{frame}[fragile]{R classes}

\begin{itemize}
\itemsep1pt\parskip0pt\parsep0pt
\item
  factor: a compact way to handle categorical data.
\end{itemize}

\begin{Shaded}
\begin{Highlighting}[]
\NormalTok{sex <-}\StringTok{ }\KeywordTok{c}\NormalTok{(}\StringTok{"M"}\NormalTok{,}\StringTok{"F"}\NormalTok{,}\StringTok{"M"}\NormalTok{,}\StringTok{"F"}\NormalTok{,}\StringTok{"F"}\NormalTok{)}
\NormalTok{sex.factor <-}\StringTok{ }\KeywordTok{as.factor}\NormalTok{(sex);}\KeywordTok{table}\NormalTok{(sex.factor)}
\NormalTok{## sex.factor}
\NormalTok{## F M }
\NormalTok{## 3 2}
\NormalTok{height <-}\StringTok{ }\KeywordTok{c}\NormalTok{(}\DecValTok{174}\NormalTok{,}\DecValTok{165}\NormalTok{,}\DecValTok{180}\NormalTok{,}\DecValTok{171}\NormalTok{,}\DecValTok{160}\NormalTok{)}
\KeywordTok{tapply}\NormalTok{(height,sex.factor,mean)}
\NormalTok{##        F        M }
\NormalTok{## 165.3333 177.0000}
\KeywordTok{gl}\NormalTok{(}\DecValTok{2}\NormalTok{,}\DecValTok{3}\NormalTok{)}
\NormalTok{## [1] 1 1 1 2 2 2}
\NormalTok{## Levels: 1 2}
\end{Highlighting}
\end{Shaded}

\end{frame}

\begin{frame}[fragile]{R classes}

\begin{itemize}
\itemsep1pt\parskip0pt\parsep0pt
\item
  list: a useful way to combine a collection of different objects.
\end{itemize}

\begin{Shaded}
\begin{Highlighting}[]
\NormalTok{family <-}\StringTok{ }\KeywordTok{list}\NormalTok{(}\DataTypeTok{name=}\StringTok{"Fred"}\NormalTok{,}\DataTypeTok{wife=}\StringTok{"Jane"}\NormalTok{,}
               \DataTypeTok{children=}\KeywordTok{c}\NormalTok{(}\StringTok{"XY"}\NormalTok{,}\StringTok{"XX"}\NormalTok{))}
\NormalTok{family$name;family[[}\DecValTok{3}\NormalTok{]][}\DecValTok{2}\NormalTok{];family[[}\StringTok{"wife"}\NormalTok{]]}
\NormalTok{## [1] "Fred"}
\NormalTok{## [1] "XX"}
\NormalTok{## [1] "Jane"}
\NormalTok{family[}\DecValTok{1}\NormalTok{:}\DecValTok{2}\NormalTok{]}
\NormalTok{## $name}
\NormalTok{## [1] "Fred"}
\NormalTok{## }
\NormalTok{## $wife}
\NormalTok{## [1] "Jane"}
\KeywordTok{unlist}\NormalTok{(family)}
\NormalTok{##      name      wife children1 children2 }
\NormalTok{##    "Fred"    "Jane"      "XY"      "XX"}
\end{Highlighting}
\end{Shaded}

\end{frame}

\begin{frame}[fragile]{R classes}

\begin{itemize}
\itemsep1pt\parskip0pt\parsep0pt
\item
  data frame: a specific list of vectors and/or factors of the same
  length.
\end{itemize}

\begin{Shaded}
\begin{Highlighting}[]
 \NormalTok{df<-}\KeywordTok{data.frame}\NormalTok{(}
\DataTypeTok{Name=}\KeywordTok{c}\NormalTok{(}\StringTok{"Alice"}\NormalTok{, }\StringTok{"Becka"}\NormalTok{, }\StringTok{"James"}\NormalTok{, }\StringTok{"Jeffrey"}\NormalTok{, }\StringTok{"John"}\NormalTok{),}
\DataTypeTok{Sex=}\KeywordTok{c}\NormalTok{(}\StringTok{"F"}\NormalTok{, }\StringTok{"F"}\NormalTok{, }\StringTok{"M"}\NormalTok{, }\StringTok{"M"}\NormalTok{, }\StringTok{"M"}\NormalTok{),}
\DataTypeTok{Age=}\KeywordTok{c}\NormalTok{(}\DecValTok{13}\NormalTok{, }\DecValTok{13}\NormalTok{, }\DecValTok{12}\NormalTok{, }\DecValTok{13}\NormalTok{, }\DecValTok{12}\NormalTok{),}
\DataTypeTok{Height=}\KeywordTok{c}\NormalTok{(}\FloatTok{56.5}\NormalTok{, }\FloatTok{65.3}\NormalTok{, }\FloatTok{57.3}\NormalTok{, }\FloatTok{62.5}\NormalTok{, }\FloatTok{59.0}\NormalTok{),}
\DataTypeTok{Weight=}\KeywordTok{c}\NormalTok{(}\FloatTok{84.0}\NormalTok{, }\FloatTok{98.0}\NormalTok{, }\FloatTok{83.0}\NormalTok{, }\FloatTok{84.0}\NormalTok{, }\FloatTok{99.5}\NormalTok{)}
\NormalTok{); df}
\NormalTok{##      Name Sex Age Height Weight}
\NormalTok{## 1   Alice   F  13   56.5   84.0}
\NormalTok{## 2   Becka   F  13   65.3   98.0}
\NormalTok{## 3   James   M  12   57.3   83.0}
\NormalTok{## 4 Jeffrey   M  13   62.5   84.0}
\NormalTok{## 5    John   M  12   59.0   99.5}
\end{Highlighting}
\end{Shaded}

\end{frame}

\begin{frame}[fragile]{R statements}

\begin{Shaded}
\begin{Highlighting}[]
\NormalTok{if(cond)\{expr\}}
\NormalTok{else if(cond)\{expr\}}
\NormalTok{else\{expr\}}

\NormalTok{for(var in seq)\{expr\}}

\NormalTok{while(cond)\{expr\}}

\NormalTok{break;next;}
\NormalTok{repeat \{expr\}}
\end{Highlighting}
\end{Shaded}

\begin{Shaded}
\begin{Highlighting}[]
\NormalTok{switch(}\DecValTok{3}\NormalTok{,}\DecValTok{1}\NormalTok{,}\DecValTok{2}\NormalTok{,}\DecValTok{3}\NormalTok{,}\DecValTok{4}\NormalTok{)}
\NormalTok{## [1] 3}
\NormalTok{switch(}\StringTok{"mean"}\NormalTok{,}\DataTypeTok{mean=}\KeywordTok{mean}\NormalTok{(}\KeywordTok{c}\NormalTok{(}\DecValTok{1}\NormalTok{,}\DecValTok{3}\NormalTok{,}\DecValTok{2}\NormalTok{)),}\DataTypeTok{median=}\DecValTok{3}\NormalTok{)}
\NormalTok{## [1] 2}
\end{Highlighting}
\end{Shaded}

\end{frame}

\begin{frame}[fragile]{R functions and libraries}

\begin{Shaded}
\begin{Highlighting}[]
\KeywordTok{install.packages}\NormalTok{(packname)}
\KeywordTok{library}\NormalTok{(packname)}

\NormalTok{funname <-}\StringTok{ }\NormalTok{function(arg1,arg2,...)\{}
  \NormalTok{statements}
  \KeywordTok{return}\NormalTok{(objects)}
\NormalTok{\}}
\end{Highlighting}
\end{Shaded}

\begin{Shaded}
\begin{Highlighting}[]
\NormalTok{centre <-}\StringTok{ }\NormalTok{function(x, type)\{}
  \NormalTok{switch(type,}
         \DataTypeTok{mean =} \KeywordTok{mean}\NormalTok{(x),}
         \DataTypeTok{median =} \KeywordTok{median}\NormalTok{(x))}
\NormalTok{\}}
\NormalTok{x <-}\StringTok{ }\KeywordTok{rcauchy}\NormalTok{(}\DecValTok{10}\NormalTok{)}
\KeywordTok{centre}\NormalTok{(x, }\StringTok{"mean"}\NormalTok{)}
\NormalTok{## [1] -7.086307}
\end{Highlighting}
\end{Shaded}

\end{frame}

\begin{frame}[fragile]{Basic graph}

\begin{Shaded}
\begin{Highlighting}[]
\CommentTok{# Basic Scatterplot Matrix}
\KeywordTok{pairs}\NormalTok{(~mpg+disp+drat+wt,}\DataTypeTok{data=}\NormalTok{mtcars)}
\end{Highlighting}
\end{Shaded}

\includegraphics{./Intro2RandStats_files/figure-beamer/unnamed-chunk-11-1.pdf}

\end{frame}

\begin{frame}[fragile]{Basic graph}

\begin{Shaded}
\begin{Highlighting}[]
\KeywordTok{boxplot}\NormalTok{(len~supp*dose, }\DataTypeTok{data=}\NormalTok{ToothGrowth, }\DataTypeTok{notch=}\OtherTok{FALSE}\NormalTok{, }
  \DataTypeTok{col=}\NormalTok{(}\KeywordTok{c}\NormalTok{(}\StringTok{"gold"}\NormalTok{,}\StringTok{"darkgreen"}\NormalTok{)))}
\end{Highlighting}
\end{Shaded}

\includegraphics{./Intro2RandStats_files/figure-beamer/unnamed-chunk-12-1.pdf}

\end{frame}

\begin{frame}[fragile]{ggplot}

\begin{Shaded}
\begin{Highlighting}[]
\KeywordTok{library}\NormalTok{(ggplot2)}
\KeywordTok{qplot}\NormalTok{(wt, mpg, }\DataTypeTok{data=}\NormalTok{mtcars, }\DataTypeTok{geom=}\KeywordTok{c}\NormalTok{(}\StringTok{"point"}\NormalTok{, }\StringTok{"smooth"}\NormalTok{), }
   \DataTypeTok{method=}\StringTok{"lm"}\NormalTok{, }\DataTypeTok{formula=}\NormalTok{y~x, }\DataTypeTok{color=}\NormalTok{cyl)}
\end{Highlighting}
\end{Shaded}

\includegraphics{./Intro2RandStats_files/figure-beamer/unnamed-chunk-13-1.pdf}

\end{frame}

\section{Intro 2 Stats.}\label{intro-2-stats.}

\begin{frame}{Content}

\begin{itemize}
\itemsep1pt\parskip0pt\parsep0pt
\item
  Bayesian Stats\\ 1.the Bayes Rule\\ 2.prior\\ 3.Hierachical Bayes\\
\item
  Frequentist Stats\\ 1.Large sample theory for the MLE\\ 2.p-value, FDR
\end{itemize}

\end{frame}

\begin{frame}{Bayesian Stats}

\begin{itemize}
\itemsep1pt\parskip0pt\parsep0pt
\item
  Bayes Rule \[Pr(\theta|Data) \varpropto Pr(\theta)Pr(Data|\theta)\]
\item
  prior\\
\end{itemize}

\begin{enumerate}
\def\labelenumi{\arabic{enumi}.}
\itemsep1pt\parskip0pt\parsep0pt
\item
  Conjugate prior distribution\\
\item
  Jeffreys prior\\
\end{enumerate}

\begin{itemize}
\itemsep1pt\parskip0pt\parsep0pt
\item
  Hierachical Bayes and an example.
\end{itemize}

\end{frame}

\begin{frame}{Frequentist Stats}

\begin{itemize}
\itemsep1pt\parskip0pt\parsep0pt
\item
  Large sample theory for the MLE\\
\item
  p-value\\
\item
  FDR
\end{itemize}

\end{frame}

\begin{frame}{Reference}

\begin{itemize}
\itemsep1pt\parskip0pt\parsep0pt
\item
  Peter Dalgaard, \emph{Introductory Statistics with R}
\item
  Winston Wang, \emph{R Graphics Cookbook}
\item
  \url{http://www.statmethods.net/}\\
\item
  Kevin P.Murphy, \emph{Machine Learning: A Probabilistic Perspective}
\end{itemize}

\end{frame}

\end{document}
